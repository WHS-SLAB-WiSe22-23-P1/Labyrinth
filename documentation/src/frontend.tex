Dieser Abschnitt beschreibt den Entwicklungsprozess, welchen wir im Laufe des Projektes durchlaufen haben, um ein komplettes Spiel in Processing zu erstellen.

\subsection{Erste Schritte in Processing}\label{subsec:erste-schritte}
Da unsere Gruppe erst wenig Erfahrung mit Processing hatte, haben wir zunächst mit dem Framework herumgespielt. Erstellt wurden einzelne kurze Programme, welche Funktionen wie dem Darstellen mehrere Quadrate in Processing erklärten.

\subsection{Umsetzung in 2D}\label{subsec:umsetzung-in-2D}
Nachdem wir genug Grundwissen über Processing hatten, haben wir mit der Entwicklung des Spiels in 2D angefangen. Dies hatte den Vorteil, dass wir die Grundmechaniken in einer einfacheren Umgebung umsetzen konnten, bevor wir in 3D übergingen.

Java ist eine objektorientierte Sprache. Deshalb wurde der Aufbau der Dateien auch objektorientiert geplant. Insgesamt wurden drei zusätzliche Klassen, zur Hauptklasse, erstellt. Eine Klasse kümmert sich um die Logik und Darstellung des Labyrinths. Eine Klasse kümmert sich um die Steuerung und Darstellung des Spielcharakters und die letzte Klasse, hilft bei der Berechnung, ob eine Kollision zwischen dem Charakter und dem Labyrinth auftritt. Gesteuert wird dies alles von der Hauptklasse, welche auch die Instanz von Processing besitzt.

Da es am Anfang noch keine funktionierende Schnittstelle zu unserer Backend und den damit verbundenen Generator für Labyrinthe gab, haben wir ein Labyrinth mit komplett zufälliger Verteilung generiert.
Mit diesem simplen Aufbau, konnten wir schon mit der Umsetzung der einzelnen Klassen beginnen. Erst wurde das Labyrinth dargestellt, dann wurde der Spielcharakter hinzugefügt. Nachdem eine einfache Steuerung in alle Richtungen funktionierte, haben wir uns um die Kollision gekümmert, damit man nicht einfach durch die Wände laufen kann.
\newline BILD von random Laby, ganz früh in der Entwicklung \newline

Als nächstes wurde die erste Schnittstelle für generierte Labyrinthe hinzugefügt, ein einfaches Auslesen einer JSON Datei, welche das Labyrinth in einem Grid gespeichert hat. 
Da die meisten Mechaniken schon zuvor umgesetzt wurden, war das Spiel damit in dem ersten komplett spielbaren Stand.

\subsection{Umsetzung in 3D}\label{subsec:umsetzung-in-3D}
Nachdem das Spiel in 2D funktionierte, sind wir auf 3D umgestiegen. Da, wie zuvor schon erwähnt, Java eine objektorientierte Sprache ist, haben wir möglichst viel Code der 2D Klassen vererbt und im Nachhinein angepasst. 
Das Umsetzen in 3D war für uns besonders interessant, da wir bislang noch nichts in 3D gemacht haben. Deshalb haben wir wieder zunächst die Dokumentationen und Tutorials von Processing durchlaufen, um zu verstehen, wie genau Processing 3D umsetzt. Es stellte sich heraus, dass das Erstellen von Körper in 3D vergleichsweise einfach war, nachdem wir vieles von der 2D Umsetzung übernehmen konnten. Jedoch muss man für die Kamera sehr viel selbst berechnen, da einem nur wenig abgenommen wird. Generell war die Kamerasteuerung einer der größten Hürden in der Umsetzung der 3D Ansichten.
Es wurden 2 Ansichten umgesetzt, wobei wir erst uns auf eine Top-Down Ansicht fokussiert haben. Dies lag unter anderem daran, dass die Steuerung des Spielcharakters der der 2D Ansicht ähnelte. Bei einer Top-Down Ansicht, sieht man ähnlich, wie bei einer 2D Ansicht von oben auf das Spielfeld. Teilweise wird dies auch gerne als Vogelperspektive bezeichnet, da die Kamera und damit der Spieler wie ein Vogel von oben schaut. 
\newline BILD 3D TOPDOWN \newline

Als die Top-Down 3D Ansicht unseren Wünschen entsprach, erstellten wir auch schon die ersten Anfänge einer First-Person 3D Ansicht. Mit einer First-Person Ansicht meint man eine Ansicht, welche die Perspektive des Spielers selbst besitzt. Im Falle unseres Labyrinthes bedeuted dies also, dass man selbst im Labyrinth steckt und nicht einfach über die Wände schauen kann. 
\newline BILD 3D First-Person \newline

Da es sicher herausstellte, dass wir dabei noch eine Hand voll mehr Hürden haben werden, fokussierten wir uns erst auf die Verschönerung der allgemeinen 3D-Ansicht. Wie man auf den Bildern erkennen kann, war die Spielwelt sehr einfarbig. Deshalb erkundigten wir uns, wie man Texturen in 3D einfügt. Dabei lernten wir, dass die Standardfunktion für einen einfachen Würfel keine Möglichkeiten für Texturen gaben. Als Lösung haben wir stattdessen eigenen Würfel mithilfe von kombinierten 2D Flächen erstellt. Diese einzelnen Flächen kann man einfach mit einer Textur versehen, die als ganzes die Wände unseres Labyrinthes darstellten.
\newline BILD 3D TopDown MIT TEXTUR \newline

Das Spiel sah nun viel schöner aus. \newline
Kommentare WIP \newline
- 3D First Person klappte\newline
- Optimierung für fast grenzenlose Labys \newline
- ALLE BILDER noch hinzufügen \newline